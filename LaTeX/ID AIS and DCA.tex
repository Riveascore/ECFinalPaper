\documentclass{umm-senior-sem}
\usepackage{cite}
\usepackage{url}

%\usepackage{setspace}
%\setstretch{2} 
\begin{document}
\title{Summary of Using Ant Colony Optimizations to Learn Fuzzy Cognitive Maps}
\author{
\alignauthor
Chris Aga \\
\affaddr{ 
University of Minnesota, Morris \\
600 East 4th St. \\	
Morris, Minnesota 56267} \\
\email{agaxx010@morris.umn.edu}
}

\maketitle

\begin{abstract}
Abstract
\end{abstract}

\keywords{Intrusion Detection Systems, Artificial Immune Systems, Negative Selection, Computer Security, Denial of Service, Danger Theory, Dendritic Cell Algorithm, Real-Time Analysis, Segmentation}

\section{Introduction}


\section{Conclusion}
\label{sec:Conclusion}
The field of intrusion detection is constantly becoming more complex and various traditional techniques are not as viable as they once used to be. There are numerous categories that intrusion detection systems are comprised of and each approach in these disparate categories have their individual strengths and weaknesses. 

New approaches such as the application of artificial immune systems are promising for intrusion detection, but the field is still young and many of its algorithms are composed of biologically naive structures that date back to the 50s~\cite{greensmith_thesis:2007}.

Recently, new immune inspired techniques such as the dendritic cell algorithm have been developed in an attempt to overcome these obstacles in intrusion detection. This algorithm has been proven to have validity in the field of intrusion detection. Also, the implementation of dynamic segmentation has been postulated to significantly improved this algorithm's results. But more improvements need to be made before the dendritic cell algorithm is fully feasible for real-time applications to everyday computing.

\nocite{*}
%^this is a very important addition so that
%references will be included property

\bibliography{Bibliography}
\bibliographystyle{abbrv}
\end{document}